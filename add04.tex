\documentclass[a4paper, 12pt]{article}
%%%%%%%%%%%%%%%%%%%%%%%%%%%%%%%%%%%%%%%%%%%%%%%%%%%%%%%%%%%%%%%%%%%%%%%%%%%%%%%%%%%%%%%%%%%%%%%%%%%%%%%%%%%%%%%%%%%%%%%%%%%%%%%%%%%%%%%%%%%%%%%%%%%%%%%%%%%%%%%%%%%%%%%%%%%%%%%%%%%%%%%%%%%%%%%%%%%%%%%%%%%%%%%%%%%%%%%%%%%%%%%%%%%%%%%%%%%%%%%%%%%%%%%%%%
% autocompile publish

\usepackage{math-hse}
\title{Дополнительные задачи}
\renewcommand{\thesubsection}{\arabic{subsection}}

\date{07.02.2020}
\begin{document}

\begin{problem}
Может ли случайная величина $X$ иметь биномиальное распределение вероятностей, если:
a) $E(X) = 6$, $D(X) =3$; b)  $E(X) = 7$, $D(X) =4$?
\end{problem}

\begin{problem}
У Пети есть два аквариума. В первом аквариуме живут $5$ гуппи и $3$ неона, 
а во втором – $3$ гуппи и $2$ неона. Петя решил поэкспериментировать и 
переселить рыбок. Он наугад берет двух рыбок из первого аквариума 
и перекладывает во второй. После этих манипуляций мы ловим одну
рыбку из второго аквариума и видим, что это – гуппи. Найдите
вероятность того, что из первого аквариума Петя переложил одну гуппи 
и одного неона.
\end{problem}

\noindent\textit{Источник к задаче 1: Е.С.Кочетков, С.О.Смерчинская. Теория вероятностей в задачах и упражнениях. Москва. 2011.}

\end{document}

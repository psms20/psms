\documentclass[a4paper, 12pt]{article}
%%%%%%%%%%%%%%%%%%%%%%%%%%%%%%%%%%%%%%%%%%%%%%%%%%%%%%%%%%%%%%%%%%%%%%%%%%%%%%%%%%%%%%%%%%%%%%%%%%%%%%%%%%%%%%%%%%%%%%%%%%%%%%%%%%%%%%%%%%%%%%%%%%%%%%%%%%%%%%%%%%%%%%%%%%%%%%%%%%%%%%%%%%%%%%%%%%%%%%%%%%%%%%%%%%%%%%%%%%%%%%%%%%%%%%%%%%%%%%%%%%%%%%%%%%
% autocompile publish

\usepackage{math-hse}
\title{Необязательное домашнее задание 1}
\renewcommand{\thesubsection}{\arabic{subsection}}
\date{к 23.01.2020}

\begin{document}

\noindent\textit{Задание не сдается на проверку, но выполнив предложенные задачи, в начале 
следующего семинара можно выйти к доске и продемонстрировать их решение.}

\begin{problem}
Вычислите:
\begin{enumerate}
\item $C_{15}^{12}$;
\item $C_8^5 + C_8^6$;
\item $\frac{13}{10}\times C_{12}^9$.
\end{enumerate}
\end{problem}

\begin{problem}
В команде по квиддичу три охотника, два загонщика, 
один ловец и один вратарь. Мадам Трюк случайным образом выбирает
трех человек из команды. С какой вероятностью среди отобранных 
будет два охотника и один загонщик? 
\end{problem}


\begin{problem}
В некоторой стране у двух кандидатов в президенты на 
текущий момент предвыборной кампании сторонников поровну. 
Случайным образом выбраны $6$ избирателей. Какова вероятность 
того, что среди них:
\begin{enumerate}
\item ни одного сторонника первого кандидата; 
\item более двух сторонников первого кандидата? [№5.11]
\end{enumerate}
\end{problem}

\noindent\textit{Источник: Макаров А.А., Пашкевич А.В. Задачник по теории вероятностей для 
студентов социально-гуманитарных специальностей. -- М.: -- МЦНМО, 2015.}

\end{document}

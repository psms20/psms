\documentclass[a4paper, 12pt]{article}
%%%%%%%%%%%%%%%%%%%%%%%%%%%%%%%%%%%%%%%%%%%%%%%%%%%%%%%%%%%%%%%%%%%%%%%%%%%%%%%%%%%%%%%%%%%%%%%%%%%%%%%%%%%%%%%%%%%%%%%%%%%%%%%%%%%%%%%%%%%%%%%%%%%%%%%%%%%%%%%%%%%%%%%%%%%%%%%%%%%%%%%%%%%%%%%%%%%%%%%%%%%%%%%%%%%%%%%%%%%%%%%%%%%%%%%%%%%%%%%%%%%%%%%%%%
% autocompile publish

\usepackage{math-hse}
\title{Случайный эксперимент. Испытания Бернулли.}
\renewcommand{\thesubsection}{\arabic{subsection}}

\date{17.01.2020}
\begin{document}

\begin{problem}
На первой полке стоят $3$~книги по истории, а на второй полке 
-- $4$~книги по теории вероятностей.
\begin{enumerate}
\item Вася сначала наугад берет одну книгу по теории вероятностей 
и одну книгу по истории. Сколько различных наборов, то есть пар 
книг он может получить?
\item Теперь представьте, что Вася решил сделать подарки своим 
однокурсникам -- подарить им эти наборы книг, но с одним дополнением: 
к подарку он может добавить шоколадку (а может не добавлять). 
Сколько различных подарочных наборов он может получить?
\end{enumerate}
\end{problem}

\begin{problem}
Правильный игральный тетраэдр подбрасывают один раз. 
Исходом такого эксперимента считается выпадение определенного 
числа очков на грани, на которую тетраэдр встает при падении.
\begin{enumerate}
\item Сколько элементарных исходов у такого эксперимента? 
Перечислите их. 
\item Является ли выпадение четного числа очков элементарным исходом? 
Обоснуйте свой ответ.
\item Правильный игральный тетраэдр подбросили $3$ 
раза. Сколько исходов у такого эксперимента? 
\item Игральный тетраэдр подбросили два раза. Найдите вероятность того, что 
в сумме за два очка выпало не менее $6$ очков. 
\end{enumerate}
\end{problem}

\begin{problem}
В группе $12$ студентов, $8$ девушек и $4$ юноши. 
Согласно традициям греческой демократии, 
в студсовет случайным образом выбирают $5$ человек. 
Найдите вероятность того, что: 
\begin{enumerate}
\item все пять выбранных будут юношами;
\item среди выбранных будет ровно одна девушка;
\item среди выбранных будет ровно две девушки;
\item среди выбранных будет хотя две девушки;
\item все пять выбранных будут девушками.
\end{enumerate}
\end{problem}

\begin{problem}
Студент-политолог для своего исследования выбрал $10$ стран, из 
них $6$ автократий, 
остальные -- демократии. Из названий стран он составил список. 
\begin{enumerate}
\item Случайным образом мы выбираем одну страну из списка. 
Можно ли считать это испытанием Бернулли? Если да, то чему 
равны вероятности успеха и неудачи?
\item Страны в списке студента упорядочены по алфавиту. 
Сколькими способами можно составить список стран, меняя их местами? 
А если из них выбрать только демократии и менять их местами? 
\item Студент случайным образом выбирает две страны из списка. 
Сколькими способами он может это сделать (считаем, что нам не важно, 
какой политический режим в выбранных странах, и то, в каком порядке 
мы выбираем страны)?
\item Студент случайным образом одновременно выбирает три страны 
из списка. С какой вероятностью среди них окажется две демократии и 
одна автократия?
\item Студент случайным образом последовательно выбирает 
несколько стран из списка (выбранные страны на каждом шаге 
вычеркиваются -- обратно в список <<не возвращаются>>). Можно 
ли считать такой эксперимент серией испытаний Бернулли? А если, 
выбрав страну, он будет записывать ее название на листок и 
<<возвращать>> обратно в список?
\end{enumerate}
\end{problem}

\begin{problem}
Правильную монетку бросают $10$~раз. Найдите вероятность 
того, что:
\begin{enumerate}
\item выпадет ровно $2$ герба;
\item выпадет ровно $5$ гербов.
\end{enumerate}
\end{problem}
\end{document}

%%% wait until conditional probability


\begin{problem}
Вычислите:
\begin{enumerate}
\item $C_{15}^{12}$
\item $C_8^5 + C_8^6$
\item $\frac{13}{10}\cdot C_{12}^9$
\end{enumerate}
\end{problem}

\begin{problem}
Колода состоит из $8$ карт --- четырех тузов и четырех королей. 
Каждый из двух игроков получает из этой колоды в закрытую по $2$ карты. 
Комбинации делятся на $2$ типа: комбинация из двух тузов считается сильной, 
а все остальные комбинации --- слабыми. 
Рассчитайте вероятности событий $A$ = <<у противника сильная комбинация>> и 
$B$ = <<у противника слабая комбинация>>, если, открыв свои карты, 
игрок увидел, что у него:
 \begin{enumerate}
 	\item Сильная комбинация;
 	\item Слабая комбинация: два короля;
 	\item Слабая комбинация: король и туз.
\end{enumerate}
\end{problem}

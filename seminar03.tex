\documentclass[a4paper, 12pt]{article}
%%%%%%%%%%%%%%%%%%%%%%%%%%%%%%%%%%%%%%%%%%%%%%%%%%%%%%%%%%%%%%%%%%%%%%%%%%%%%%%%%%%%%%%%%%%%%%%%%%%%%%%%%%%%%%%%%%%%%%%%%%%%%%%%%%%%%%%%%%%%%%%%%%%%%%%%%%%%%%%%%%%%%%%%%%%%%%%%%%%%%%%%%%%%%%%%%%%%%%%%%%%%%%%%%%%%%%%%%%%%%%%%%%%%%%%%%%%%%%%%%%%%%%%%%%
% autocompile publish

\usepackage{math-hse}
\title{Математическое ожидание и дисперсия дискретной случайной величины. 
Биномиальное распределение}
\renewcommand{\thesubsection}{\arabic{subsection}}
\usepackage{multicol}

\date{28.01.2020}
\begin{document}

\begin{problem} 
Дан ряд распределения случайной величины $X$. 

\begin{table}[ht!]
\centering
\begin{tabular}{|l|l|l|l|l|l|}
\hline
X & $-5$  & $-1$  & $0$  & $1$  & $2$   \\ \hline
p & $1/5$ &  & $1/10$ & $1/10$ & $1/5$ \\ \hline
\end{tabular}
\end{table}
\noindentНайдите математическое ожидание, дисперсию и 
стандартное отклонение случайной величины $X$.
\end{problem}

\begin{problem}
В ходе исследования проведенного среди жителей мегаполиса, 
респондентам был задан вопрос: <<Укажите, пожалуйста, общее 
количество гаджетов Apple, которыми обладаете лично вы>>. 
Распределение вероятностей случайной величины – 
числа гаджетов Apple – приведено в таблице:

$$
\begin{tabular}{|c|c|c|c|c|c|c|c|}
\hline
\text{Число продуктов Apple} & 0 & 1 & 2 & 3 & 4 & 5 & 6 \\
\hline
\text{Оценка вероятности} & 0.36 & 0.25 & 0.11 & 0.11 & ? & 0.06 & 0.03\\
\hline
\end{tabular}
$$

\begin{enumerate}
\item Укажите пропущенную вероятность.
\item Вычислите среднее число продуктов Apple у жителя мегаполиса.
\item Найдите дисперсию и стандартное отклонение. 
\item Проинтерпретируйте полученные значения. Что можно сказать про 
выраженность/невыраженность дифференциации ответов людей по 
рассмотренному вопросу?[№6.6]
\end{enumerate}
\end{problem}

\begin{problem} 
Представьте, что перед вами стоит такая задача: 
необходимо сравнить успеваемость студентов в~двух 
группах. Распределение оценок студентов в первой и второй
группах описывается следующими законами:
\begin{table}[ht!]
\centering
\begin{tabular}{|l|l|l|l|l|}
\hline
X & $2$    & $3$    & $4$    & $5$    \\ \hline
p & 0.2 & 0.3 & 0.25 & 0.25 \\ \hline
\end{tabular}
\begin{tabular}{|l|l|l|l|l|}
\hline
Y & $2$    & $3$    & $4$    & $5$   \\ \hline
p & 0.5 & 0.05 & 0.05 & 0.4 \\ \hline
\end{tabular}
\end{table}\\
У какой группы средний ожидаемый
балл выше? А в какой группе разброс оценок меньше? 
\end{problem}

\begin{problem} 
Случайные величины $X$ и~$Y$ независимы. Известно, 
что $E(X)=2$, $E(Y)=4$, $D(X)=4$, $D(Y)=9$. Найдите 
математическое ожидание и~дисперсию случайной 
величины $W$.
\begin{multicols}{2}
\begin{enumerate}
\item $W=5X+2Y$
\item $W=4X-7Y-2$
\item $W=2Y+5$
\item $W=-3X-Y+6$.
\end{enumerate}
\end{multicols}
\end{problem}

\begin{problem} 
Спидометр автомобиля определяет скорость в~километрах 
в час. Дисперсия показаний бортового компьютера равна $4$. 
Найти дисперсию и стандартное отклонение показаний скорости, 
выраженной в милях в час ($1$ миля = $1609$ м). [№6.21]
\end{problem}

\begin{problem}
Известно, что $60$\% студентов очной формы обучения совмещают 
обучение с работой. Для проведения интервью мы случайным 
образом выбираем $10$ студентов.
\begin{enumerate}
\item Какова вероятность того, что среди выбранных респондентов 
будет ровно $4$ работающих студента?
\item Какова вероятность того, что среди выбранных респондентов 
будет не менее $8$ работающих студентов?
\item Сколько работающих студентов, в среднем, мы можем встретить 
среди выбранных $10$ студентов?
\item Пусть $X$ -- число работающих студентов среди выбранных $10$ 
респондентов. Найдите дисперсию и стандартное отклонение случайной 
величины $X$.
\end{enumerate}
\end{problem}


\noindent\textit{Задачи 2 и 5 взяты из Макаров А.А., Пашкевич А.В. Задачник по теории вероятностей для 
студентов социально-гуманитарных специальностей. -- М.: -- МЦНМО, 2015.}

\end{document}

\documentclass[a4paper, 12pt]{article}
%%%%%%%%%%%%%%%%%%%%%%%%%%%%%%%%%%%%%%%%%%%%%%%%%%%%%%%%%%%%%%%%%%%%%%%%%%%%%%%%%%%%%%%%%%%%%%%%%%%%%%%%%%%%%%%%%%%%%%%%%%%%%%%%%%%%%%%%%%%%%%%%%%%%%%%%%%%%%%%%%%%%%%%%%%%%%%%%%%%%%%%%%%%%%%%%%%%%%%%%%%%%%%%%%%%%%%%%%%%%%%%%%%%%%%%%%%%%%%%%%%%%%%%%%%
% autocompile publish

\usepackage{math-hse}
\title{Необязательное домашнее задание 2}
\renewcommand{\thesubsection}{\arabic{subsection}}
\date{к 07.02.2020}

\begin{document}

\noindent\textit{Задание не сдается на проверку, но выполнив предложенные задачи, в начале 
следующего семинара можно выйти к доске и продемонстрировать их решение.}

\begin{problem}
Выпускник факультета
социальных наук послал заявку на участие с~докладом в~двух
независимых международных конференциях. На первую из них он может
попасть с~вероятностью~$0.6$, на вторую "---
с~вероятностью~$0.3$. Рассмотрите случайную величину $X$ "---
количество международных конференций, оргкомитет которых
\textit{откажет} выпускнику в~участии с~докладом. Найдите: 
\begin{enumerate}
\item закон распределения случайной величины~$X$;
\item математическое ожидание,
дисперсию и~стандартное отклонение этой случайной величины. [№6.37 в задачнике]
\end{enumerate}
\end{problem}

\begin{problem}
В~оптовом магазине минеральная вода
продается либо поштучно, либо упаковками по~$2$ или $16$~бутылок.
Предпочтения покупателей этой воды известны: вероятность покупки
одной бутылки равна~$0.74$, упаковки из двух бутылок "---
$0.24$, упаковки из $16$~бутылок "--- $0.02$. Найдите
дисперсию величины <<число бутылок в~одной покупке>>. [№6.25 в задачнике]
\end{problem}

\begin{problem}
Математическое ожидание дискретной случайной величины $X$ 
равно $-2$, а ее стандартное отклонение равно $3$. 
Математическое ожидание дискретной случайной величины $Y$ 
равно $1$, а ее стандартное отклонение равно $2$. 
Известно, что величины $X$ и $Y$ независимы. 
Найдите математическое ожидание и дисперсию случайной 
величины $U = -4X + 2Y + 5$.
\end{problem}

\noindent\textit{Источник: Макаров А.А., Пашкевич А.В. Задачник по теории вероятностей для 
студентов социально-гуманитарных специальностей. -- М.: -- МЦНМО, 2015.}


\end{document}

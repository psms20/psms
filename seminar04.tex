\documentclass[a4paper, 12pt]{article}
%%%%%%%%%%%%%%%%%%%%%%%%%%%%%%%%%%%%%%%%%%%%%%%%%%%%%%%%%%%%%%%%%%%%%%%%%%%%%%%%%%%%%%%%%%%%%%%%%%%%%%%%%%%%%%%%%%%%%%%%%%%%%%%%%%%%%%%%%%%%%%%%%%%%%%%%%%%%%%%%%%%%%%%%%%%%%%%%%%%%%%%%%%%%%%%%%%%%%%%%%%%%%%%%%%%%%%%%%%%%%%%%%%%%%%%%%%%%%%%%%%%%%%%%%%
% autocompile publish

\usepackage{math-hse}
\title{Математическое ожидание и дисперсия биномиальной случайной величины. 
Совместное распределение}
\renewcommand{\thesubsection}{\arabic{subsection}}
\newcommand{\E}{$\text{E}$}
\newcommand{\D}{$\text{D}$}

\date{07.02.2020}
\begin{document}

\begin{problem}
Известно, что $60$\% студентов очной формы обучения совмещают 
обучение с работой. Для проведения интервью мы случайным 
образом выбираем $10$ студентов.
\begin{enumerate}
\item Какова вероятность того, что среди выбранных респондентов 
будет ровно $4$ работающих студента?
\item Какова вероятность того, что среди выбранных респондентов 
будет не менее $8$ работающих студентов?
\item Сколько работающих студентов, в среднем, мы можем встретить 
среди выбранных $10$ студентов?
\item Пусть $X$ -- число работающих студентов среди выбранных $10$ 
респондентов. Найдите дисперсию и стандартное отклонение случайной 
величины $X$.
\end{enumerate}
\end{problem}

\begin{problem}
Летний вечер. Ежик и медвежонок пьют чай и собираются смотреть на звезды. 
Известно, что за ночь падает примерно $10000$ звезд. 
Вероятность увидеть падающую звезду равна $0.025$. 
\footnote{Конечно, звезды не падают, это метеоры, но так интереснее.}
Пусть случайная величина $N$ – число падающих звезд, которые увидят ежик 
с медвежонком. Найдите математическое ожидание и стандартное отклонение 
случайной величины $N$. 
\end{problem}

\begin{problem}
Случайные величины $X$ и~$Y$ независимы. Найдите 
математическое ожидание и~дисперсию случайной 
величины $V$, если известно,
что $\E(X)=1$, $\E(Y)=5$, $\D(X)=3$, $\D(Y)=4$.

\begin{enumerate}
\item $V=-6X+3Y$;
\item $V=5X-2Y-3$.
\end{enumerate}
\end{problem}

\begin{problem}
Известно совместное распределение случайных величин $X$ и $Y$. 
Каждая из этих случайных величин соответствует одному вопросу в 
некотором тесте знаний и описывает правильность ответа на него: 
\begin{table}[h!]
\centering
\begin{tabular}{|l|l|l|}
\hline
$X~\textbackslash~Y$	& 0 & 1 \\ \hline
0	& 0.3 & 0.1 \\ \hline
1	& 0.1 & 0.5 \\ \hline
\end{tabular}
\end{table}


\begin{enumerate}
\item Запишите маргинальные распределения случайных величин $X$ и $Y$.
\item Проверьте, являются ли случайные величины независимыми. 
\item Найдите условную вероятность $\text{P}(Y=1~|~X=1)$ и сравните ее с 
безусловной вероятностью $\text{P}(Y=1)$. 
\item Запишите ряд распределения числа правильных ответов на эти два вопроса -- 
суммы случайных величин. Запишите ряд распределения произведения случайных величин $X \cdot Y$.
\end{enumerate}
\end{problem}

\begin{problem}
В психологическом тесте два  вопроса имеют по три варианта 
ответа на каждый. Каждому из вариантов ответа на каждый вопрос 
присваивается сырой балл: $0$, $1$, $2$ в зависимости от выраженности 
тестируемого свойства. Совместное распределение сырых баллов за 
каждый ответ  задано таблицей:
		
\begin{table}[ht!]
\centering
\begin{tabular}{|l|l|l|l|}
	\hline
	$X~\textbackslash~Y$	& 0 & 1 & 2 \\ \hline
	0	& 0.2 & 0.05 & 0 \\ \hline
	1	& 0.15 & 0.1 & 0.05 \\ \hline
	2	& 0.05 & 0.2 & ? \\ \hline
\end{tabular}
\end{table}

\begin{enumerate}
\item Запишите маргинальные распределения случайных величин $X$ и $Y$.
\item Можно ли считать, что ответы на вопросы независимы? 
\item Найти условные вероятности $\text{P}(Y=2~|~X=2)$ и $\text{P}(Y=2~|~X=0)$. 
\item Найдите математическое ожидание и дисперсию случайной величины $X\cdot Y$.
\end{enumerate}

\end{problem}

\end{document}

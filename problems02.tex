\documentclass[a4paper, 12pt]{article}
%%%%%%%%%%%%%%%%%%%%%%%%%%%%%%%%%%%%%%%%%%%%%%%%%%%%%%%%%%%%%%%%%%%%%%%%%%%%%%%%%%%%%%%%%%%%%%%%%%%%%%%%%%%%%%%%%%%%%%%%%%%%%%%%%%%%%%%%%%%%%%%%%%%%%%%%%%%%%%%%%%%%%%%%%%%%%%%%%%%%%%%%%%%%%%%%%%%%%%%%%%%%%%%%%%%%%%%%%%%%%%%%%%%%%%%%%%%%%%%%%%%%%%%%%%
% autocompile publish

\usepackage{math-hse}
\title{Необязательное домашнее задание 2}
\renewcommand{\thesubsection}{\arabic{subsection}}
\date{к 28.01.2020}

\begin{document}

\noindent\textit{Задание не сдается на проверку, но выполнив предложенные задачи, в начале 
следующего семинара можно выйти к доске и продемонстрировать их решение.}


\begin{problem}
Рассмотрим случайную величину $X$ – число баллов за тест, в котором $6$ вопросов 
и на каждый вопрос $5$ вариантов ответа. За правильный ответ начисляется $1$ очко, 
за неправильный – $0$. Ответы выбираются наугад. 
Постройте закон распределения вероятностей $X$ и 
найдите её математическое ожидание. 
\footnote{На основе №6.26 из \textit{Макаров А.А., Пашкевич А.В. Задачник по теории вероятностей для 
студентов социально-гуманитарных специальностей. -- М.: -- МЦНМО, 2015.}}
\end{problem}

\begin{problem}
Игральный кубик бросают два раза. Случайная величина $X$ – число чётных 
чисел, выпавших за эти два броска. Постройте ряд распределения случайной величины
$X$ и найдите её математическое ожидание.
\end{problem}

\begin{problem}
К восхождению на вершину горы независимо друг от друга 
приступили две группы (экспедиции). Вероятность неудачного
восхождения для первой группы составляет $0.8$, для второй
$0.6$. Рассмотрите случайную величину $X$ – количество экспедиций, 
которые смогут подняться на вершину горы. Найдите закон распределения 
случайной величины $X$ и её математическое ожидание.
\footnote{№6.39, там же.}
\end{problem}


\end{document}

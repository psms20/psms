\documentclass[a4paper, 12pt]{article}
%%%%%%%%%%%%%%%%%%%%%%%%%%%%%%%%%%%%%%%%%%%%%%%%%%%%%%%%%%%%%%%%%%%%%%%%%%%%%%%%%%%%%%%%%%%%%%%%%%%%%%%%%%%%%%%%%%%%%%%%%%%%%%%%%%%%%%%%%%%%%%%%%%%%%%%%%%%%%%%%%%%%%%%%%%%%%%%%%%%%%%%%%%%%%%%%%%%%%%%%%%%%%%%%%%%%%%%%%%%%%%%%%%%%%%%%%%%%%%%%%%%%%%%%%%
% autocompile publish

\usepackage{math-hse}
\title{Дополнительные задачи}
\renewcommand{\thesubsection}{\arabic{subsection}}

\date{28.01.2020}
\begin{document}

\begin{problem}
Найдите пропущенное значение случайной величины $X$, 
если известно, что оно положительно и что $D(X)=2$.
\begin{table}[ht!]
\centering
\begin{tabular}{|l|l|l|}
\hline
X & $-1$            &               \\ \hline
p & $1/3$ & $2/3$ \\ \hline
\end{tabular}
\end{table}
\end{problem}

\begin{problem}
Случайная величина $X$ принимает три значения: $-2$, $0$, $2$. 
Составить её ряд распределения, если $E(X) = 0$, $D(X) = 2$. 
\end{problem}

\begin{problem}
Игральную кость бросают $6$ раз. Вычислите математическое 
ожидание числа попарно различных цифр.
\end{problem}

\noindent\textit{Источник: Е.С.Кочетков, С.О.Смерчинская. Теория вероятностей в задачах и упражнениях. Москва. 2011.}

\end{document}

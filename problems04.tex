\documentclass[a4paper, 12pt]{article}
%%%%%%%%%%%%%%%%%%%%%%%%%%%%%%%%%%%%%%%%%%%%%%%%%%%%%%%%%%%%%%%%%%%%%%%%%%%%%%%%%%%%%%%%%%%%%%%%%%%%%%%%%%%%%%%%%%%%%%%%%%%%%%%%%%%%%%%%%%%%%%%%%%%%%%%%%%%%%%%%%%%%%%%%%%%%%%%%%%%%%%%%%%%%%%%%%%%%%%%%%%%%%%%%%%%%%%%%%%%%%%%%%%%%%%%%%%%%%%%%%%%%%%%%%%
% autocompile publish

\usepackage{math-hse}
\title{Необязательное домашнее задание 4}
\renewcommand{\thesubsection}{\arabic{subsection}}
\date{к 14.02.2020}

\begin{document}

\begin{problem}
В психологическом тесте два  вопроса имеют по три варианта 
ответа на каждый. Каждому из вариантов ответа на каждый вопрос 
присваивается сырой балл: $0$, $1$, $2$ в зависимости от выраженности 
тестируемого свойства. Совместное распределение сырых баллов за 
каждый ответ  задано таблицей:
		
\begin{table}[ht!]
\centering
\begin{tabular}{|l|l|l|l|}
	\hline
	$X~\textbackslash~Y$	& 0 & 1 & 2 \\ \hline
	0	& 0.2 & 0.05 & 0 \\ \hline
	1	& 0.15 & 0.1 & 0.05 \\ \hline
	2	& 0.05 & 0.2 & ? \\ \hline
\end{tabular}
\end{table}

\begin{enumerate}
\item Запишите маргинальные распределения случайных величин $X$ и $Y$.
\item Можно ли считать, что ответы на вопросы независимы? 
\item Найти условные вероятности $\text{P}(Y=2~|~X=2)$ и $\text{P}(Y=2~|~X=0)$. 
\item Найдите математическое ожидание и дисперсию случайной величины $X\cdot Y$.
\end{enumerate}
\end{problem}

\begin{problem}
В случайном экперименте независимые случайные величины $X$ и $Y$ 
заданы рядами распределений:

\begin{center}
\begin{tabular}{|l|l|l|}
\hline
X & $1$ & $2$\\
\hline
p & 0.1 & 0.9 \\
\hline
\end{tabular} \hspace{1cm}
\begin{tabular}{|l|l|l|l|}
\hline
Y & $-1$ & $0$ & $1$\\
\hline
p & 0.2 & 0.3 & 0.5 \\
\hline
\end{tabular}
\end{center}

Постройте таблицу их совместного распределения.

\end{problem}


\noindent\textit{Источник: Макаров А.А., Пашкевич А.В. Задачник по теории вероятностей для 
студентов социально-гуманитарных специальностей. -- М.: -- МЦНМО, 2015.}


\end{document}
